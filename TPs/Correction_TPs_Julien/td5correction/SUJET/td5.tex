\documentclass[a4paper,11pt]{article}
\usepackage{graphicx, subfigure}
\usepackage{amssymb,amstext,amsmath, array,textpos, url, hyperref, enumerate, listings, colortbl }
\usepackage{chngpage}
\usepackage[T1]{fontenc}
\usepackage[utf8]{inputenc}
\usepackage{lmodern}
\usepackage{algorithmic, algorithm}
\usepackage[french]{babel}
\usepackage{xcolor}
\definecolor{graybg}{gray}{0.85}
\definecolor{graybg2}{gray}{0.90}
\definecolor{graybg3}{gray}{0.95}
\pagestyle{plain}
\bibliographystyle{alpha}
\usepackage{verbatim}

\def\thesection       {\Roman{section}}
\def\thesubsection       {\arabic{subsection}}


\lstset{extendedchars=false}
\lstset{language=c}
\lstset{% general command to set parameter(s)
		basicstyle=\scriptsize,          % print whole listing small
		keywordstyle=\color{blue}\bfseries,
		frameround=tttt,
		commentstyle=\color{red},
		frame=single,
		numbers=left,
		stringstyle=\ttfamily,      % typewriter type for strings
			showstringspaces=false}     % no special string spaces
			%/LISTINGS

% Exercice number
\newcounter{numexos}%Création d'un compteur qui s`appelle numexos
\setcounter{numexos}{0}%initialisation du compteur
\newcommand{\exercice}[1]%
{%Création d'une macro ayant un paramètre
  \addtocounter{numexos}{1}%chaque fois que cette macro est appelée, elle ajoute 1 au compteur numexos
  \paragraph{Q.\thenumexos:}%
	{#1} %Met en rouge Exercice et la valeur du compteur appelée par \thenumeexos
}

%Correction
\newcommand{\sectioncorrection}[1]%
{
\ifx\tpcorrection \undefined %
\else %
\begin{center}
\textcolor{red}{Correction} \\
\fcolorbox{black}{graybg}{
\begin{minipage}{0.9\textwidth}
	{#1} 
\end{minipage}
}
\end{center}
\fi
}

% Guillemets
\newcommand{\ofg}[1]{\og{#1\fg{}}}

% Correction

% comment out above definition of todo

\newboolean{reponse}
\setboolean{reponse}{true}
\ifthenelse{\boolean{reponse}}
{
  \newenvironment{correction}
  {\color{red} \small}
  {\color{black} \normalsize}
}{
\newsavebox{\trashcan}
\newenvironment{correction}
  {\begin{lrbox}{\trashcan}}
  {\end{lrbox}}
}

\newenvironment{tip}
	{	\fbox\{\begin{minipage}{1\textwidth} \includegraphics[width=0.8cm]{../tips.png}}
	{	\end{minipage}\} }
%  {\vspace{10pt} \begin{fbox} \begin{minipage}{1\textwidth} \includegraphics[width=0.8cm]{../tips.png}}
%  {\end{minipage}\end{fbox}}

\begin{document}

\changepage{3cm}%amount added to textheight
{1cm}%amount added to textwidth
{-1cm}%amount added to evensidemargin
{-1cm}%amount added to oddsidemargin
{}%amount added to columnsep
{-2cm}%amount added to topmargin
{}%amount added to headheight
{}%amount added to headsep
{}%amount added to footskip



\vspace{0.1\textheight}

\begin{tabular}{m{0.4\textwidth}m{0.6\textwidth}}

  \begin{center}
    \includegraphics[width=5.5cm]{uvsq-logo-cmjn.jpg}
  \end{center}

  &

  \begin{center}
 	\LARGE{\textbf{M2 - Compilation Avanc\'{e}e}} \\
	\large{(COA 2016-2017)} \\
	\Huge{TD5} \\%\large{(1 s\'eance) }\\
	\Large{Dominance et Post-dominance} \\
	\large{~}\\
    	\small{
	\url{hugo.brunie.ocre@cea.fr}\\
	\url{julien.jaeger@cea.fr}\\
    	\url{patrick.carribault@cea.fr}}\\

  \end{center}
  \\
\end{tabular}


\section{Post-dominance it\'er\'ee et warnings}
Dans le TP pr\'ec\'edent, nous avons calcul\'e et affich\'e la fronti\`ere de post-dominance avec nos propres structures.
Dans ce TPs, nous allons nous servir des structures de gcc pour calculer la fronti\`ere de post-dominance et la fronti\`ere de post-dominance it\'er\'ee afin d'afficher un warning sur les n\oe{}uds provoquant un probl\`eme potentiel.

%%%%%%%%%%%%%%%%%%%%%%
%%%%% QUESTION 1 %%%%%
%%%%%%%%%%%%%%%%%%%%%%
\exercice{Chercher la fonction \textit{compute\_dominance\_frontiers} dans les sources de gcc. Utiliser les bitmaps pour calculer et afficher la fronti\`ere de post-dominance de chaque basic block. 
}
%%%%%%%%%%%%%%%%%%%%%%
%%%% CORRECTION 1 %%%%
%%%%%%%%%%%%%%%%%%%%%%
\sectioncorrection{Voir fichier \textbf{plugin\_TP5\_1.cpp}. }




%%%%%%%%%%%%%%%%%%%%%%
%%%%% QUESTION 2 %%%%%
%%%%%%%%%%%%%%%%%%%%%%
\exercice{A partir de ces PDFs, calculer la PDF d'un ensemble de n\oe{}uds.}

%%%%%%%%%%%%%%%%%%%%%%
%%%% CORRECTION 2 %%%%
%%%%%%%%%%%%%%%%%%%%%%
\sectioncorrection{Voir fichier \textbf{plugin\_TP5\_2.cpp}. }



%%%%%%%%%%%%%%%%%%%%%%
%%%%% QUESTION 3 %%%%%
%%%%%%%%%%%%%%%%%%%%%%
\exercice{Apr\`s avoir v\'erifi\'e la PDF de l'ensemble des n\oe{}uds avec un appel MPI, afficher un message pour dire si il y a un probl\`eme ou non.}

%%%%%%%%%%%%%%%%%%%%%%
%%%% CORRECTION  3 %%%
%%%%%%%%%%%%%%%%%%%%%%
\sectioncorrection{Voir fichier \textbf{plugin\_TP5\_3.cpp}}




%%%%%%%%%%%%%%%%%%%%%%
%%%%% QUESTION 4 %%%%%
%%%%%%%%%%%%%%%%%%%%%%
\exercice{A partir de cette derni\`ere PDF, calculer la PDF it\'er\'ee. Pour cela, il faut remonter l'arbre des BBs selon leur PDF jusqu\`a arriver \`a un \'etat stable.}


%%%%%%%%%%%%%%%%%%%%%%
%%%% CORRECTION 4 %%%%
%%%%%%%%%%%%%%%%%%%%%%
\sectioncorrection{Voir fichier \textbf{plugin\_TP5\_3.cpp}}



%%%%%%%%%%%%%%%%%%%%%%
%%%%% QUESTION 5 %%%%%
%%%%%%%%%%%%%%%%%%%%%%
\exercice{A partir de cette PDF it\'er\'ee, afficher les basic blocks et les lignes de codes g\'en\'erant le probl\`eme potentiel.}


%%%%%%%%%%%%%%%%%%%%%%
%%%% CORRECTION 5 %%%%
%%%%%%%%%%%%%%%%%%%%%%
\sectioncorrection{Voir fichier \textbf{plugin\_TP5\_5.cpp}}


\end{document}
