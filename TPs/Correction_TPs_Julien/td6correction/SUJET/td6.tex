\documentclass[a4paper,11pt]{article}
\usepackage{graphicx, subfigure}
\usepackage{amssymb,amstext,amsmath, array,textpos, url, hyperref, enumerate, listings, colortbl }
\usepackage{chngpage}
\usepackage[T1]{fontenc}
\usepackage[utf8]{inputenc}
\usepackage{lmodern}
\usepackage{algorithmic, algorithm}
\usepackage[french]{babel}
\usepackage{xcolor}
\definecolor{graybg}{gray}{0.85}
\definecolor{graybg2}{gray}{0.90}
\definecolor{graybg3}{gray}{0.95}
\pagestyle{plain}
\bibliographystyle{alpha}
\usepackage{verbatim}

\def\thesection       {\Roman{section}}
\def\thesubsection       {\arabic{subsection}}


\lstset{extendedchars=false}
\lstset{language=c}
\lstset{% general command to set parameter(s)
		basicstyle=\scriptsize,          % print whole listing small
		keywordstyle=\color{blue}\bfseries,
		frameround=tttt,
		commentstyle=\color{red},
		frame=single,
		numbers=left,
		stringstyle=\ttfamily,      % typewriter type for strings
			showstringspaces=false}     % no special string spaces
			%/LISTINGS

% Exercice number
\newcounter{numexos}%Création d'un compteur qui s'appelle numexos
\setcounter{numexos}{0}%initialisation du compteur
\newcommand{\exercice}[1]%
{%Création d'une macro ayant un paramètre
  \addtocounter{numexos}{1}%chaque fois que cette macro est appelée, elle ajoute 1 au compteur numexos
  \paragraph{Q.\thenumexos:}%
	{#1} %Met en rouge Exercice et la valeur du compteur appelée par \thenumeexos
}

%Correction
\newcommand{\sectioncorrection}[1]%
{
\ifx\tpcorrection \undefined %
\else %
\begin{center}
\textcolor{red}{Correction} \\
\fcolorbox{black}{graybg}{
\begin{minipage}{0.9\textwidth}
	{#1} 
\end{minipage}
}
\end{center}
\fi
}

% Guillemets
\newcommand{\ofg}[1]{\og{#1\fg{}}}

% Correction

% comment out above definition of todo

\newboolean{reponse}
\setboolean{reponse}{true}
\ifthenelse{\boolean{reponse}}
{
  \newenvironment{correction}
  {\color{red} \small}
  {\color{black} \normalsize}
}{
\newsavebox{\trashcan}
\newenvironment{correction}
  {\begin{lrbox}{\trashcan}}
  {\end{lrbox}}
}

\newenvironment{tip}
	{	\fbox\{\begin{minipage}{1\textwidth} \includegraphics[width=0.8cm]{../tips.png}}
	{	\end{minipage}\} }
%  {\vspace{10pt} \begin{fbox} \begin{minipage}{1\textwidth} \includegraphics[width=0.8cm]{../tips.png}}
%  {\end{minipage}\end{fbox}}

\begin{document}

\changepage{3cm}%amount added to textheight
{1cm}%amount added to textwidth
{-1cm}%amount added to evensidemargin
{-1cm}%amount added to oddsidemargin
{}%amount added to columnsep
{-2cm}%amount added to topmargin
{}%amount added to headheight
{}%amount added to headsep
{}%amount added to footskip


\vspace{0.1\textheight}

\begin{tabular}{m{0.4\textwidth}m{0.6\textwidth}}

  \begin{center}
    \includegraphics[width=5.5cm]{uvsq-logo-cmjn.jpg}
  \end{center}

  &

  \begin{center}
 	\LARGE{\textbf{M2 - Compilation Avanc\'{e}e}} \\
	\large{(CPA 2016-2017)} \\
	\Huge{TD6} \\
	\Large{Cr\'eation d'une nouvelle directive} \\
	\Large{avec plugin GCC} \\
	\large{~}\\
    	\small{
	\url{hugo.brunie.ocre@cea.fr}\\
	\url{julien.jaeger@cea.fr}\\
    	\url{patrick.carribault@cea.fr}}\\

  \end{center}
  \\
\end{tabular}


\section{D\'eclaration d'une simple directive}

Le but de cette partie est de r\'ecup\'erer les informations contenues dans la
directive \`a partir d'un plugin GCC.

%%%%%%%%%%%%%%%%%%%%%%
%%%%% QUESTION 1 %%%%%
%%%%%%%%%%%%%%%%%%%%%%
\exercice{En vous aidant de la documentation sur les plugins GCC, d\'eclarer une
nouvelle directive. Cette directive servira \`a d\'efinir le nom d'une fonction sur
laquelle nous souhaiterons appliquer un traitement sp\'ecial \`a partir d'une nouvelle passe. 
Dans un premier temps et \`a partir de la fonction pass\'ee \`a la fonction
\textit{c\_register\_pragma}, affichez uniquement un message lorsque le nouveau pragma
est reconnu.

Cette directive sera utilis\'ee de la mani\`ere suivante dans le code qui l'exploitera :\\
\texttt{\#pragma instrument function ma\_fonction}
}

%%%%%%%%%%%%%%%%%%%%%%
%%%% CORRECTION 1 %%%%
%%%%%%%%%%%%%%%%%%%%%%
\sectioncorrection{Voir fichier \textbf{plugin\_TP56\_1.cpp}. }



%%%%%%%%%%%%%%%%%%%%%%
%%%%% QUESTION 2 %%%%%
%%%%%%%%%%%%%%%%%%%%%%
\exercice{En vous inspirant de la fonction \textit{handle\_pragma\_target} pr\'esent dans le fichier
\texttt{c-family/c-pragma.c}, afficher uniquement le nom de la fonction annot\'ee lorsque l'argument du pragma
est de type \texttt{CPP\_NAME}. 
}

%%%%%%%%%%%%%%%%%%%%%%
%%%% CORRECTION 2 %%%%
%%%%%%%%%%%%%%%%%%%%%%
\sectioncorrection{Voir fichier \textbf{plugin\_TP56\_2.cpp}. }


%%%%%%%%%%%%%%%%%%%%%%
%%%%% QUESTION 3 %%%%%
%%%%%%%%%%%%%%%%%%%%%%
\exercice{\'Etender maintenant cette directive pour g\'erer plusieurs noms de fonction.
Toujours en vous inspirant de la fonction \textit{handle\_pragma\_target}, parser et afficher ces noms.  

Cette directive pourrait \^etre utilis\'ee de la mani\`ere suivante dans le code qui l'exploitera :\\
\texttt{\#pragma instrument function(f1,f2,f3,f4)}
}

%%%%%%%%%%%%%%%%%%%%%%
%%%% CORRECTION 3 %%%%
%%%%%%%%%%%%%%%%%%%%%%
\sectioncorrection{On commence par v\'erifier si le premier token est une parenth\`ese ouvrante.
Ensuite on affiche le premier nom. Si on a rencontr\'e une parenth\`ese ouvrante, alors il peut y avoir une liste
de nom: on lit les virgules de s\'eparations. Une fois la liste finie, on doit trouver une parenth\`ese fermante.
On renvoie un message d'erreur si le format n'est pas valide (\texttt{\#pragma instrumente function NAME} ou 
\texttt{\#pragma instrumente function (NAME1,NAME2,...)} ). Voir fichier \textbf{plugin\_TP6\_3.cpp}. }


%%%%%%%%%%%%%%%%%%%%%%
%%%%% QUESTION 4 %%%%%
%%%%%%%%%%%%%%%%%%%%%%
\exercice{Toujours en vous inspirant du fichier \texttt{c-family/c-pragma.c}, rajouter un test
dans le traitement de la directive, afin de v\'erifier qu'elle a bien \'et\'e plac\'ee \`a l'ext\'erieur 
d'une fonction. Retournez une erreur dans le cas contraire.
}

%%%%%%%%%%%%%%%%%%%%%%
%%%% CORRECTION 4 %%%%
%%%%%%%%%%%%%%%%%%%%%%
\sectioncorrection{Voir fichier \textbf{plugin\_TP56\_4.cpp}. }


\section{Interaction avec une passe}

Nous souhaitons maintenant que notre directive puisse interagir avec une passe.
Pour ce faire, nous allons enregistrer les noms des fonctions pass\'ees par la directive
dans une liste d\'efinie en variable globale du plugin.

%%%%%%%%%%%%%%%%%%%%%%
%%%%% QUESTION 5 %%%%%
%%%%%%%%%%%%%%%%%%%%%%
\exercice{Construire cette liste en utilisant la structure vecteur d\'efinie dans le fichier
\texttt{vec.h} des sources GCC. Afficher le contenu de ce vecteur apr\`es chaque pragma.}


%%%%%%%%%%%%%%%%%%%%%%
%%%% CORRECTION 5 %%%%
%%%%%%%%%%%%%%%%%%%%%%
\sectioncorrection{D\'eclaration d'un vecteur global de \texttt{char *}. Utilisation de \textit{safe\_push}
pour ins\'erer les noms dans le vecteur. Voir fichier \textbf{plugin\_TP56\_5.cpp}. }



%%%%%%%%%%%%%%%%%%%%%%
%%%%% QUESTION 6 %%%%%
%%%%%%%%%%%%%%%%%%%%%%
\exercice{Pour le moment, certains noms pr\'esents dans des pragmas non-valides sont ins\'er\'es dans la liste.
N'ins\'erer que les noms pr\'esents dans les pragmas valides. V\'erifier qu'un nom de fonction n'a pas \'et\'e utilis\'e plus d'une
fois dans la directive. Si c'est le cas, afficher un warning.}

%%%%%%%%%%%%%%%%%%%%%%
%%%% CORRECTION 6 %%%%
%%%%%%%%%%%%%%%%%%%%%%
\sectioncorrection{D\'eclaration d'un vecteur temporaire de \texttt{char *}. On ne reverse les noms de ce vecteur dans le vecteur global
que si le pragma a \'et\'e valid\'e. On profite de la copie pour v\'erifier que le nom de la fonction n'est pas d\'ej\`a pr\'sente dans le vecteur.
Voir fichier \textbf{plugin\_TP56\_6.cpp}. }


%%%%%%%%%%%%%%%%%%%%%%
%%%%% QUESTION 7 %%%%%
%%%%%%%%%%%%%%%%%%%%%%
\exercice{En vous aidant des TDs pr\'ec\'edents, cr\'eer une passe qui affiche le nom de
chaque fonction. Modifiez la gate pour que cette passe soit activ\'ee seulement si le
nom de la fonction est contenue dans la liste globale.}


%%%%%%%%%%%%%%%%%%%%%%
%%%% CORRECTION 7 %%%%
%%%%%%%%%%%%%%%%%%%%%%
\sectioncorrection{On reprend ce qu'on a fait dans le TD2 q3. Seul changement: la fonction gate ne renvoie plus \texttt{true}. 
On r\'ecup\`re le nom de la fonction courante sour forme de \texttt{char *}, et on renvoie la valeur de la fonction \textit{already\_pushed}
de la question pr\'ec\'edente.
Voir fichier \textbf{plugin\_TP56\_7.cpp}. }



%%%%%%%%%%%%%%%%%%%%%%
%%%%% QUESTION 8 %%%%%
%%%%%%%%%%%%%%%%%%%%%%
\exercice{En quittant GCC, v\'erifiez que tous les noms de fonction contenus dans la
liste globale ont \'et\'e utilis\'es. Dans le cas contraire, affichez un warning sur les noms
restants. Il s'agit probablement de fonctions non-d\'efinies dans le code annot\'e.

%%%%%%%%%%%%%%%%%%%%%%
%%%% CORRECTION 8 %%%%
%%%%%%%%%%%%%%%%%%%%%%
\sectioncorrection{On enregistre un nouveau callback avec en argument de position \texttt{PLUGIN\_FINISH} et la fonction \textit{plugin\_finalize}.
On modifie la gate pour enlever du vecteur les noms de fonctions trouv\'ees \`a la compilation.
Dans la fonction \textit{plugin\_finalize}, on v\'erifie la taille du vecteur global. Si il n'est pas nul, alors il reste des arguments
que l'on a pas rencontr\'e. On les affiche. 
Voir fichier \textbf{plugin\_TP56\_8.cpp}. }




}


\end{document}
